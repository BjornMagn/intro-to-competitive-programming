\documentclass[8pt,a4paper,landscape,oneside]{amsart}
\usepackage{amsmath, amsthm, amssymb, amsfonts}
\usepackage[T1]{fontenc}
\usepackage[utf8]{inputenc}
\usepackage{booktabs}
\usepackage{caption}
\usepackage{color}
\usepackage{fancyhdr}
\usepackage{float}
\usepackage{fullpage}
\usepackage{subcaption}
\usepackage[scaled]{beramono}
\usepackage{titling}
\usepackage{datetime}
\usepackage{enumitem}
\usepackage{multicol}

% Minted
\usepackage{minted}
\newcommand{\code}[1]{\inputminted[fontsize=\normalsize,baselinestretch=1]{cpp}{code/#1}}

\pagestyle{fancy}
\lhead{Lund University}
\rhead{\thepage}
\cfoot{}
\setlength{\headheight}{15.2pt}
\setlength{\droptitle}{-20pt}
\posttitle{\par\end{center}}
\renewcommand{\headrulewidth}{0.4pt}
\renewcommand{\footrulewidth}{0.4pt}

\begin{document}

\begin{multicols*}{3}
%\maketitle
\thispagestyle{fancy}
\vspace{-3em}

\tableofcontents
\section{Code Templates}
    \subsection{Java Template}
        A Java template.
        \code{template.java}
    \subsection{Python Template}
        A Python template
        \code{template.py}
    \subsection{C++ Template}
        A C$++$ template
        \code{template.cpp}
    \subsection{Fast IO Java}
        Kattio with easier names
        \code{fastio.java}

\section{Data Structures}
    \subsection{Binary Indexed Tree}
        Also called a fenwick tree. Builds in O(n log n), answers what is the sum from 0 to i in O(log n) and updates an element in O(log n).
        \code{DS/BIT.java}
    \subsection{Segment Tree}
        More general than a fenwick tree. Can adapt other operations than sum, e.g. min and max.
        \code{DS/ST.java}
    \subsection{Union Find}
        This data structure is used in varoius algorithms, for example Kruskals algorithm for finding an Minimal Spanning Tree in a weighted graph. Also it can be used for backward simulation of dividing a set.
        \code{DS/UF.java}
\end{multicols*}
\end{document}
