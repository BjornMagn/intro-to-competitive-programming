\documentclass[8pt,a4paper,landscape,oneside]{amsart}
\usepackage{amsmath, amsthm, amssymb, amsfonts}
\usepackage[T1]{fontenc}
\usepackage[utf8]{inputenc}
\usepackage{booktabs}
\usepackage{caption}
\usepackage{color}
\usepackage{fancyhdr}
\usepackage{float}
\usepackage{fullpage}
\usepackage{subcaption}
\usepackage[scaled]{beramono}
\usepackage{titling}
\usepackage{datetime}
\usepackage{enumitem}
\usepackage{multicol}
\usepackage{bm}

% Minted
\usepackage{minted}
\newcommand{\code}[1]{\inputminted[fontsize=\normalsize,baselinestretch=1]{cpp}{code/#1}}

\pagestyle{fancy}
\lhead{Lund University}
\rhead{\thepage}
\cfoot{}
\setlength{\headheight}{15.2pt}
\setlength{\droptitle}{-20pt}
\posttitle{\par\end{center}}
\renewcommand{\headrulewidth}{0.4pt}
\renewcommand{\footrulewidth}{0.4pt}

\newcommand{\bigO}{\mathcal{O}}

\begin{document}

\begin{multicols*}{3}
%\maketitle
\thispagestyle{fancy}
\vspace{-3em}

\tableofcontents
\section{Code Templates}
    \subsection{Java Template}
        A Java template.
        \code{template.java}
    \subsection{Python Template}
        A Python template
        \code{template.py}
    \subsection{C++ Template}
        A C$++$ template
        \code{template.cpp}
    \subsection{Fast IO Java}
        Kattio with easier names
        \code{fastio.java}

\section{Data Structures}
    \subsection{Binary Indexed Tree}
        Also called a fenwick tree. Builds in $\bigO(n \log{n})$ from an array. Querry sum from 0 to i in $\bigO(\log{n})$ and updates an element in $\bigO(\log{n})$.
        \code{DS/BIT.java}
    \subsection{Segment Tree}
        More general than a fenwick tree. Can adapt other operations than sum, e.g.\ min and max.
        \code{DS/ST.java}
    \subsection{Lazy Segment Tree}
        More general implementation of a segment tree where its possible to increase whole segments by some diff, with lazy propagation. Implemented with arrays instead of nodes, which probably has less overhead to write during a competition.
        \code{DS/LazySegmentTree.java}
    \subsection{Union Find}
        This data structure is used in varoius algorithms, for example Kruskals algorithm for finding a Minimal Spanning Tree in a weighted graph. Also it can be used for backward simulation of dividing a set.
        \code{DS/UF.java}
    \subsection{Monotone Queue}
        Used in sliding window algorithms where one would like to find the minimum in each interval of a given length. Amortized $\bigO(n)$ to find min in each of these intervals in an array of length $n$. Can easily be used to find the maximum as well.
        \code{DS/MinMonQue.java}
\section{Graph Algorithms}
    \subsection{Djikstras algorithm}
        Finds the shortest distance between two Nodes in a weighted graph in $\bigO (|E| \log{|V|})$ time.
        \code{Graphs/Djikstra.java}
    \subsection{Bipartite Graphs}
        The Hopcroft-Karp algorithm finds the maximal matching in a bipartite graph. Also, this matching can together with Könings theorem be used to construct a minimal vertex-cover, which as we all know is the complement of a maximum independent set. Runs in $\bigO (|E|\sqrt{|V|})$. 
        \code{Graphs/BiGraph.java}
\section{Dynamic Programming}
    \subsection{Longest Increasing Subsequence}
        Finds the longest increasing subsequence in an array in $\bigO(n \log{n})$ time. Can easility be transformed to longest decreasing/nondecreasing/nonincreasing subsequence.
        \code{DP/lis.java}
    \subsection{Knuuth Morris Pratt substring}
        Finds if $\bm{w}$ is a subarray to $\bm{s}$ in linear time. 
        \code{DP/kmp.java}
\section{Etc}
    \subsection{System of Equations}
        Solves the system of equations $\bm{A}\bm{x} = \bm{b}$ by Gaussian elimination. This can for example be used to determine the expected value of each node in a markov chain. Runns in $\bigO (N^3)$.
        \code{Etc/SolveSystem.java}
\end{multicols*}
\end{document}
